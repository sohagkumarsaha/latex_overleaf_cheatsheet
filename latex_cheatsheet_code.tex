Cheatsheet latex overleaf: 

% Table automatic numbering  
Table \ref{table1}




% Reference numbering
\cite{bib1}





% Itemize List
\begin{itemize}
	\item Item1
	\item Item2
	\item Item3
\end{itemize}

%  Numered list 
\begin{enumerate}
	\item Item1 
	\item Item2
	\item Item3 
end{enumerate}


% refering sections
Section~\ref{sec:background}

\section{Background Study}
\label{sec:background}


% Refering figure number
\ref{fig1} 


% Figure
\begin{figure*}[t!]
  \centering
  \includegraphics[width=0.80\linewidth]{images/figure1.png}
  \caption{Statistical feature selection stacking framework.}
  \label{fig1}
\end{figure*}



% Equation numbering
Eq. ~\eqref{eq:logistic1}

\begin{equation}
    f(z) = \frac{1}{1 + e^{-z}}
    \label{eq:logistic1}
\end{equation}



% Sample Table

% Table 1
\begin{table}[htbp]
\centering
\caption{Major Risk Factors Behind CVD}
\label{table1}
\begin{tabular}{cc}
\Xhline{2\arrayrulewidth} % Creates a double thick line
\makecell{Controllable Factors} & \makecell{Non-controllable Factors} \\
\hline
Obesity                 &       Age                  \\
Stress                  &       Gender               \\
Eating habits           &       Genetics             \\
Sedentary lifestyle     &       Family history       \\
Smoking                 &       Ethnicity            \\
\Xhline{2\arrayrulewidth} % Creates a double thick line
\end{tabular}
\end{table}




% Another table reference 

\ref{dataset_summary_table}.


\begin{table}[ht]
    \centering
    \caption{Dataset summary.}
    \label{dataset_summary_table}
    \begin{tabular}{p{0.20\linewidth}| p{0.40\linewidth}| p{0.25\linewidth}}
    \hline \hline
      \textbf{Datasets}  & \textbf{Sources} & \textbf{Descriptions} 
        \\ \hline

        Dataset 1 & Framingham Dataset & 15 features \newline 1-Target variable \newline Categorical: 7 \newline Continuous: 8 \\
        \hline
        Dataset 2 & UCI Heart Disease Dataset (Combination of Cleveland, Hungary, Switzerland, and the VA Long Beach databases) & 13 features \newline 1-Target variable \newline Categorical: 6 \newline Continuous: 7 \\
        \hline
        Dataset 3 & The Z-Alizadeh Saini Dataset & 53 Features \newline 1-Target variable \newline Categorical: 34 \newline Continuous: 19 \\
        
        \hline 
    \end{tabular}
    \label{tab:my_label}
\end{table}




% Algorithm refering
\ref{alg:StackingSVM}


% algorithm 1 
\begin{algorithm}
\caption{Stacking with SVM as Meta Model}
\label{alg:StackingSVM}
\begin{algorithmic}[1]
\Require Training data TD
\Ensure An ensemble stacking model M
\State \textbf{Step 1:} Train base models on the entire training set
\For{$k \leftarrow 1$ to $K$}
    \State Train a model $M_k$ from TD
\EndFor
\State \textbf{Step 2:} Create a training set for the meta-model
\For{$x_i$ belonging to TD}
    \State Obtain a value $\{m_i, y_i\}$, where $m_i = \{m_1(x_i), m_2(x_i), \ldots, m_K(x_i)\}$
\EndFor
\State \textbf{Step 3:} Learn a meta-model
\State Train a new model $m'$ from the pool of $\{m_i, y_i\}$
\State \textbf{Step 4:} Return the ensemble classifier M
\State $M(x) = m'(m_1(x), m_2(x), \ldots, m_K(x))$
\end{algorithmic}
\end{algorithm}



% Citations and Equations coloring

% color and hyper link in references 
\usepackage{xcolor}
\usepackage{hyperref}


% Set color for links
\hypersetup{
    colorlinks=true,
    linkcolor=blue, % blue  % magenta
    filecolor=magenta,      
    urlcolor=cyan,
    citecolor=blue % blue % red
}


%% adding color to the brackets in the references or citations 
\makeatletter
\renewcommand{\@cite}[2]{\textcolor{blue}{[{#1\if@tempswa , #2\fi}]}}
\makeatother


% Adding color to the brackets of equation number
\makeatletter
\renewcommand{\eqref}[1]{\textcolor{blue}{(\ref{#1})}}
\makeatother





% deleting initial  indent
\noindent



% Reference auto numbering mode in IEEE templates


%% for new reference style of auto numbering
\bibliographystyle{unsrt}




% Add the end before document ended
\bibliographystyle{ieeetr}
\bibliography{reference.bib}


% images in svg and images in pdf or png or jpg format 




% for svg import some important libraries first
\usepackage{graphicx}
\graphicspath{ {./images/} }
\usepackage{svg}

% then write the following code to import images inn .svg format

% image import code for .svg file
 \begin{figure*}[htbp]
  \centering
  \includesvg[width=1.00\linewidth]{images/MG_EMS.svg}
  \caption{Categorization of microgrid based on operation, control, config, scenarios, location, and type of sources}
  \label{microgrid_category}
\end{figure*}



%image import code for .png or .pdf file 
\begin{figure*}[t!]
  \centering
  \includegraphics[width=1.00\linewidth]{images/MG_EMS.pdf}
  \caption{Categorization of microgrid based on operation, control, config, scenarios, location, and type of sources}
  \label{microgrid_category}
\end{figure*}





% An table formatting code (example) 


% Table for comparison of widely used optimization algorithms 

% Table spanning both columns
\begin{table*}[htbp]
\renewcommand\arraystretch{1.3}
\centering
\caption{Comparative Analysis of widely used optimization algorithms in  published works \cite{allwyn2023comprehensive, chopra2022golden}}
\begin{tabularx}{\textwidth}{ m{3.75cm}  m{6cm}   m{7cm}  }
%\hline
\Xhline{2\arrayrulewidth} % Creates a double thick line
\textbf{Name of Algorithm} & \textbf{Key Strengths} & \textbf{Shortcomings} \\ \hline
Linear Programming & 
Reliable and faster in terms of computational efficiency  & 
The majority of problems in the real-world are nonlinear, and developing a linear model for them can lead to significant losses \cite{geem2001new}. \\ \hline

Mixed Integer Linear Programming & 
Guaranteed convergence of the solution with global optimum values \cite{urbanucci2018limits}. Fast and effective solvers available commercially. & 
Optimum solution cannot be obtained if the constraints are contradictory \cite{urbanucci2018limits} and non-linear effects cannot be accounted. \\ \hline

Non-linear Programming & 
Results are more reliable as the model exactly replicates the nonlinear characteristics of the system under study. & 
Optimal results cannot be obtained if the functions used in computation are not differentiable. Requirement for feasible starting point to obtain global optima. \cite{geem2001new}. \\ \hline

Dynamic Programming & 
The sensitivity analysis can be inferred from problem solution\cite{nelson1968dynamic}, formulation can be done based on initial value problems \cite{bellman1965dynamic}. & 
Increase in the number of variables increases exponentially with increasing number of functions requiring more memory space \cite{geem2001new}. \\ 

\Xhline{2\arrayrulewidth} % Creates a double thick line
%\hline

\end{tabularx}
\label{table:optimization_algorithms}
\end{table*}






